
\documentclass{l3proj}
\begin{document}
\title{A Social Network for Project Task Management}
\author{Gustavo Almansa\\
        Darren Burns \\
        Leonardo Linhares \\
        Euan Parker \\
        Joao Vitor Sousa \\
	Tom Wallis \\
}
\date{13 October 2014}
\maketitle
\begin{abstract}

The abstract goes here

\end{abstract}
\educationalconsent
\tableofcontents
%==============================================================================
\chapter{Introduction}
\label{intro}

During the first week of the Team Project, each member researched about several
task management applications available. The advantages and disadvantages of each
one are described below.

\section{Asana}
\label{asana}

Asana is a task management software to organise a team's project which main
feature is the email independence. The software consists of a web-based app and
a mobile app for Android and iOS. Asana is very powerful, with good integration
to services such as Dropbox, GitHub, Chrome, Evernote and Wordpress, however, it
is complex and it requires time to learn and understand all its features. The
web app has a non-friendly interface and can be confusing. The main problem is
the communication among project’s members, because there is no chat, only a
comment area inside each ticket or task. Despite its features, many of them in
common to others task management softwares, Asana still has a dependence with
Skype or email to be more useful.

\section{Trello}
\label{trello}

Trello is a web-based task management platform which employs the Japanese 'Kanban'
system, originally developed by Toyota. Kanban uses a hierarchical system of
cards, contained within lists, which themselves are contained within boards, to
manage how projects are managed. Interestingly, Kanban is often used for
supermarket inventories and managing the flows of tasks in the physical world --
it isn't just a method to abstract out a project, but is a way of managing any
production system.
Trello connects to IFTTT (an automation platform which ties together various APIs)
and Zapier, which automates processes based on events on an array of online
platforms. It has an API with several open-source wrappers in several languages.
Trello is used to manage dataflow in law cases, lesson planning, and so on, as
well as project management, due to Kanban's versatility.
In some ways, Trello can be considered an implementation of Kanban (or 'eKanban')
and secondarily a project management platform; nevertheless Trello claims to be
used by organisations as large as Microsoft, Google, The New York Times and
Adobe. 


\section{Producteev}
\label{producteev}

%==============================================================================


\section{Bugzilla}
\label{Bugzilla}

Bugzilla is a bug tracker and testing tool created and used by Mozila,
and while not an actual version control system it still has some
interesting features which could be useful.

It integrates with SVN, Git, CVS and Bazaar.  Some interesting
features include automated 'reminders', whereby when bugs are
reported but left unresolved the system can be configured to send
regular e-mails to specified people informing them of these bugs.
This system is known as the 'whine' system.  

When any change is made to a known bug, this change is added to an
automated bug report, the format of which is:\\
Bug ID || Description || Component\footnote{The part of software the
  bug is in.} || Current Status

Bugzilla integrates with e-mail, although this seems like it could be
redundant when integrating with IM.  
Although comparatively simple to use, criticisms of Bugzilla have been
made that it is hard to manage errors and there exist many bugs in the
program itself.  



\chapter{Design}
\label{design}

The following diagrams (especially figure \ref{fig:alice}) illustrate the
process...

%==============================================================================
\chapter{Implementation}
\label{impl}

In this chapter, we describe how the implemented the system.

%------------------------------------------------------------------------------
\section{User Interface}

Blah blah blah
Blah blah blah
Blah blah blah
Blah blah blah

% - - - - - - - - - - - - - - - - - - - - - - - - - - - - - - - - - - - - - - -
\subsection{Foo}

Blah blah blah
Blah blah blah
Blah blah blah
Blah blah blah

%------------------------------------------------------------------------------
\section{Database Model}

\begin{enumerate}
\item Blah blah blah
\item Blah blah blah
\item Blah blah blah
\item Blah blah blah
\end{enumerate}



%==============================================================================
\chapter{Evaluation}

We evaluated the project by...

%==============================================================================
\chapter{Conclusion}

A great project!

%==============================================================================
\section{Contributions}

Here we explain that Lewis Carroll wrote chapter \ref{intro}. John Wayne
was out riding his horse every day and didn't do anything. Marilyn Monroe
was great at getting the requirements specification and coordinating the
writing of the report. Betty Davis did the coding of the kernel of the
project, described in Chapter \ref{impl}.  James Dean handled the
multimedia content of the project.

%==============================================================================
\bibliographystyle{plain}
\bibliography{example}
\end{document}
