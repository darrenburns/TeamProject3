\pdfoutput=1

\documentclass{l4proj}

%
% put any packages here
%

\begin{document}
\title{Social Networking with Project Task Management}
\author{Euan Parker}
\date{2015/2016}
\maketitle

\begin{abstract}
deontobot!
\end{abstract}

\educationalconsent
%
%NOTE: if you include the educationalconsent (above) and your project is graded an A then
%      it may be entered in the CS Hall of Fame
%
\tableofcontents
%==============================================================================

\chapter{Introduction}
\pagenumbering{arabic}

\section{First Section in Chapter}
The quick brown fox jumped over the lazy dog.
The quick brown fox jumped over the lazy dog.
The quick brown fox jumped over the lazy dog.
The quick brown fox jumped over the lazy dog.
The quick brown fox jumped over the lazy dog \cite{Powers}.
The quick brown fox jumped over the lazy dog.
The quick brown fox jumped over the lazy dog.
The quick brown fox jumped over the lazy dog.
The quick brown fox jumped over the lazy dog.

\subsection{A subsection}
The quick brown fox jumped over the lazy dog.
The quick brown fox jumped over the lazy dog.
The quick brown fox jumped over the lazy dog.
The quick brown fox jumped over the lazy dog.

The quick brown fox jumped over the lazy dog.
The quick brown fox jumped over the lazy dog.
The quick brown fox jumped over the lazy dog.
The quick brown fox \cite{Wyner} jumped over the lazy dog.
The quick brown fox jumped over the lazy dog.

\chapter{Related Work}
A number of works were studied that related to the issue of team communication and team management.

\section{Architectures, Coordination, and Distance: Conway's Law and Beyond}

In this paper it is discussed how the architecture of a system being developed often dictates how the team organisation will be split up.  It states that this alone is not sufficient to fully coordinate a large development team, and taking into account different tasks, milestones, plans and processes must be considered to effectively manage a team.  It is said however that this is rarely successfully achieved in the real world, due to unpredictable factors in the development process such as inaccurate estimates, changing requirements, incomplete solutions and personnel leaving or joining the project.  

These problems are circumvented largely due to informal communication and organisation, whereby the developers will discuss components of their work while at lunch, on a break or over social media.

A case study was carried out on a technology department of a multinational country, which was chosen for the factor that sites in different countries had the added organisational complication of multiple languages.  The department was currently working on a project whereby different teams worked on different components of a large system, and simulated the rest of the system for the purpose of development.  It was found however that in creating the simulated components the problem description lacked certain essential details and as a result each team had made their own assumptions regarding missing information.  As a result each team ended up with different assumptions and when the components were integrated they did not work correctly.  It was found that team members were reluctant to record these assumptions as they viewed it as taking away from time that could be spent developing the system, when they had already discussed it within their teams.

\section{Global Software Engineering: The Future of Socio-technical Coordination}

This paper discusses the current trend of software development projects to have multiple teams in different locations, often even different continents.  This is due to the increased ease of communication over long distances as well as the accessibility of skilled workers regardless of where they live.

It is found however that many of the usual aspects of co-located projects are lost in a distributed team, whereby through frequent, informal interactions team members get to know the skills and expertise of their team members.  Ideas are also more effectively communicated when teams share a common native language as there is less chance of misunderstandings, or ideas that are difficult to communicate.

As a result globalisation is considered by some to be one of the major research challenges in requirements engineering.  Suggested solutions were improving communications and more disciplined project management such as assigning roles and frequently documenting what has been done


\section{Designing Task Visualizations to Support the Coordination of Work in Software Development}

This paper discusses the fact that the software development tools are primarily focused on assisting the technical side of development, despite the fact that large parts of the process are organisational tasks.

It is discussed that software development originally had the form whereby all code would be developed in one place, then over time it would be maintained by other people in different locations, thus communication between developers could be done via comments in the code.  Now however this has evolved such that it is now commonplace to have distributed development teams and a variety of communication methods to support the development and maintenance of systems.  With small local teams physically walking into someones office is often the best way to discuss something, but this is not possible for larger scale companies.

Another issue identified is the visualisation of the development of software.  It is proposed that visualising data gathered from version control could be used to identify trends in the evolution of very large software systems, for example to identify candidates for refactoring.  

a case study made was one whereby participants of a project were represented by a coloured hexagon.  The colour represented the status of the person in relation to a task (e.g. in progress, completed or not started)
The hexagons were arranged according to their position within the organisation.  This allowed for an overview of the participants to be easily viewed as well as what tasks were completed.






%\vspace{-7mm}
\begin{figure}
\centering
%\includegraphics[height=9.2cm,width=13.2cm]{uroboros.pdf}
\vspace{-30mm}
\caption{An alternative hierarchy of the algorithms.}
\label{uroborus}
\end{figure}

The quick brown fox jumped over the lazy dog.
The quick brown fox jumped over the lazy dog.
The quick brown fox jumped over the lazy dog.
The quick brown fox jumped over \cite{ckt} the lazy dog.
The quick brown fox jumped over the lazy dog.
The quick brown fox jumped over the lazy dog.
The quick brown fox jumped over the lazy dog.
The quick brown fox jumped over the lazy dog.

\section{The Lazy Dog}
The quick brown fox jumped over the lazy dog.
The quick brown fox jumped over the lazy dog.
The quick brown fox jumped over the lazy dog.

The quick brown fox jumped over the lazy dog.
The quick brown fox \cite{am97} jumped over the lazy dog.
The quick brown fox jumped over the lazy dog.
The quick brown fox jumped over the lazy dog.
The quick brown fox jumped over the lazy dog.
The quick brown fox jumped over the lazy dog.

%%%%%%%%%%%%%%%%
%              %
%  APPENDICES  %
%              %
%%%%%%%%%%%%%%%%
\begin{appendices}

\chapter{Running the Programs}
An example of running from the command line is as follows:
\begin{verbatim}
\end{verbatim}


\chapter{Generating Random Graphs}
\label{sec:randomGraph}

\begin{verbatim}
\end{verbatim}
\end{appendices}

%%%%%%%%%%%%%%%%%%%%
%   BIBLIOGRAPHY   %
%%%%%%%%%%%%%%%%%%%%

\bibliographystyle{plain}
\bibliography{bib}

\end{document}
