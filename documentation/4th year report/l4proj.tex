\pdfoutput=1

\documentclass{l4proj}

%
% put any packages here
%

\begin{document}
\title{Social Networking with Project Task Management}
\author{Euan Parker}
\date{2015/2016}
\maketitle

\begin{abstract}

\end{abstract}

\educationalconsent
%
%NOTE: if you include the educationalconsent (above) and your project is graded an A then
%      it may be entered in the CS Hall of Fame
%
\tableofcontents
%==============================================================================

\chapter{Introduction}
\pagenumbering{arabic}

\chapter{Related Work}
A number of works were studied that related to the issue of team communication and team management.

\section{Architectures, Coordination, and Distance: Conway's Law and Beyond}

In this paper it is discussed how the architecture of a system being developed often dictates how the team organisation will be split up.  It states that this alone is not sufficient to fully coordinate a large development team, and taking into account different tasks, milestones, plans and processes must be considered to effectively manage a team.  It is said however that this is rarely successfully achieved in the real world, due to unpredictable factors in the development process such as inaccurate estimates, changing requirements, incomplete solutions and personnel leaving or joining the project.  

These problems are circumvented largely due to informal communication and organisation, whereby the developers will discuss components of their work while at lunch, on a break or over social media.

A case study was carried out on a technology department of a multinational country, which was chosen for the factor that sites in different countries had the added organisational complication of multiple languages.  The department was currently working on a project whereby different teams worked on different components of a large system, and simulated the rest of the system for the purpose of development.  It was found however that in creating the simulated components the problem description lacked certain essential details and as a result each team had made their own assumptions regarding missing information.  As a result each team ended up with different assumptions and when the components were integrated they did not work correctly.  It was found that team members were reluctant to record these assumptions as they viewed it as taking away from time that could be spent developing the system, when they had already discussed it within their teams.

\section{Global Software Engineering: The Future of Socio-technical Coordination}

This paper discusses the current trend of software development projects to have multiple teams in different locations, often even different continents.  This is due to the increased ease of communication over long distances as well as the accessibility of skilled workers regardless of where they live.

It is found however that many of the usual aspects of co-located projects are lost in a distributed team, whereby through frequent, informal interactions team members get to know the skills and expertise of their team members.  Ideas are also more effectively communicated when teams share a common native language as there is less chance of misunderstandings, or ideas that are difficult to communicate.

As a result globalisation is considered by some to be one of the major research challenges in requirements engineering.  Suggested solutions were improving communications and more disciplined project management such as assigning roles and frequently documenting what has been done


\section{Designing Task Visualizations to Support the Coordination of Work in Software Development}

This paper discusses the fact that the software development tools are primarily focused on assisting the technical side of development, despite the fact that large parts of the process are organisational tasks.

It is discussed that software development originally had the form whereby all code would be developed in one place, then over time it would be maintained by other people in different locations, thus communication between developers could be done via comments in the code.  Now however this has evolved such that it is now commonplace to have distributed development teams and a variety of communication methods to support the development and maintenance of systems.  With small local teams physically walking into someones office is often the best way to discuss something, but this is not possible for larger scale companies.

Another issue identified is the visualisation of the development of software.  It is proposed that visualising data gathered from version control could be used to identify trends in the evolution of very large software systems, for example to identify candidates for refactoring.  

a case study made was one whereby participants of a project were represented by a coloured hexagon.  The colour represented the status of the person in relation to a task (e.g. in progress, completed or not started)
The hexagons were arranged according to their position within the organisation.  This allowed for an overview of the participants to be easily viewed as well as what tasks were completed.


\section Who Should Fix This Bug?

This paper suggests a semi-automated way of assigning bug reports to a developer, through use of a machine learning algorithm.  This algorithm looks at an open bug repository in order to learn the kind of reports each different developer on the team resolves.  When a new bug report arrives the the system will then recommend a group of developers that could be expected to resolve this based on what they have done previously.

\section Sysiphus: Enabling informal collaboration in global software development

In this paper it is discussed that projects with a globally distributed development team informal communication is severely limited, resulting in difficulty in speading tactic knowledge.  This knowledge is what allows for a full overview of an organisation and thus all participants do not have access to the rational behind decisions.  As a result when problems arise it can take days to rectify despite the issues being easily rectifiable if the relevant stakeholders were identified.

Thus the paper proposes a system that captures information about organisational roles and structures it for use.  The system encourages participants to make communication and issues explicite in the system models and thus relevant stakeholders are clear.

\section Virtual software team project management

This paper looks at the necessary differences in organising a virtual team (i.e. geographically distributed) versus organising a collocated team.  Geographic, temporal, cultural and linguistic distance all hinder coordination, cooperation and communication. 

It is stated that global projects carry additional risk of delay or failure due to linguistic and cultural difference, motivational differences and temporal distance.  The paper proposes how best to organise a virtual team in order to prevent this.  It is essential to formalise roles, relationships and rules to aid in communication and control.  This process must be documented and formalised in order to ensure these are followed. 

\section Using Developer Activity Data to Enhance Awareness during Collaborative Software Development

This discusses that with the increase in scale and distribution of software development, the shared understanding that developers would previously have had is more difficult to attain, as this originally would have come about through formal and informal face-to-face meetings and gatherings.  Thus a model is proposed that seeks to encapsulate the ordering of tasks, artefacts and developers relevant to particular work contexts in projects.  This model would be built up using data gathered from the developers interactions with IDEs.  Specifically it would identify entities (developers, tasks, artefacts), associated with a particular work context, then identify relationships among these. Furthermore some information can be acquired regarding where developers work, and when.

\section Hipikat: A Project Memory for Software Management

This paper discusses how as a result of distributed development teams new members of such companies cannot learn from their experiences colleagues, due to absence of informal encounters.
 the paper proposes a solution through a tool that provides a "project memory".  The tool helps newcomers working on a task find similar examples in the source code from previous work done.  

\section Collaboration in Global Software Projects at Siemens: An Experience Report

In this paper it is discussed that in distributed development teams it is inevitable that as a result of so many different developers in so many different places a huge knowledge base will be created but is largely inaccessible.  This paper proposes a system to gather this information and compiling it in a form that is more accessible.

This information was gathered through semi-structured interviews whereby participants filled in a form describing a practice they employed in dealing with a Global Software Development related issue that had been successful.

\section A Practical Management and Engineering Approach to Offshore Collaboration

\section Social Coding in GitHub: Transparency and Collaboration in an Open Software Repository

This paper discusses how from looking at activity information in github a large amount of information can be inferred pertaining to the persons technical aims and vision regarding a piece of code.  Moreover it can be used to guess a projects prospective success based on how regularly it is updated, and how supported a project is based on attention from the community.

The information was obtained through semi-structured interviews on GitHub users, in order to document and understand how different aspects of GitHubs functionality were used.  The users were asked to walk through a session on GitHub and describe how they interpreted information displayed on the site.

\section CVS Integration with Notification and Chat: Lightweight Software Team Collaboration

It is discussed how Concurrent Version Systems (CVS) can be used in organisation of software development, but it is proposed that these do not encapsulate the informal discussions regarding the code and thus could stand to be augmented by the addition of a lightweight event notification system.  This tool would display messages from the CVS and offer a chat system for developers to discuss them, thus allowing for a combination of formal documentation and informal discussion.

\section Unifying Artifacts and Activities in a Visual Tool for Distributed Software Development Teams

This paper proposes a tool that combines the two sources of complexity in software development - Code complexity and the complexity that comes with producing it.  This tool would provide a development environment in which both the code would be visible as well as activity that has recently been performed in relation to the code.

\section What is Chat Doing in the Workplace?








%%%%%%%%%%%%%%%%
%              %
%  APPENDICES  %
%              %
%%%%%%%%%%%%%%%%
\begin{appendices}


\end{appendices}

%%%%%%%%%%%%%%%%%%%%
%   BIBLIOGRAPHY   %
%%%%%%%%%%%%%%%%%%%%

\bibliographystyle{plain}
\bibliography{bib}

\end{document}
